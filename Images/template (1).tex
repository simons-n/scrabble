%% I suggest using this file as a template for your homework solutions

%% you can use percentage sign (%) to comment things out of a .tex document
%% template for homework solutions
\documentclass[12pt]{article}

\usepackage{amssymb}

\pagestyle{empty}

%% page layout
\setlength{\topmargin}{-25mm}
\addtolength{\textheight}{3cm}
\setlength{\textwidth}{17cm}
\addtolength{\oddsidemargin}{-2.1cm}
\setlength{\parindent}{1cm}
\def\baselinestretch{1.2}

%% user defined commands
\newcommand{\qed}{~\Box}

\begin{document}

%% you can edit the following to produce a suitable header for your document
\begin{tabbing}
Mathematics 241/0 \hspace{9cm} \= {\bf Name:} Pete Brooksbank \\
Spring Semester, 2016 \>  {\bf Week 0, Set A} 
\end{tabbing}
\vspace*{7mm}

Here is a short example that illustrates how to use \LaTeX~ to present a mathematical proof.
This is a lovely proof by Euclid of the infinitude of the prime numbers, and
uses a ``proof technique" that we will encounter shortly in the course.
%% you can force breaks of varying amounts in displayed text using the following command
\vspace*{5mm}

%% the mathematical environment is opened and closed using the dollar sign ($)
%% use this environment for any sort of mathematical expression. 
\noindent {\bf Problem 1.} {\em Prove that there are infinitely many prime numbers.}
%% use the following command to space out your text
\vspace*{3mm}

\noindent {\bf Solution:}  We give the usual ``proof by contradiction".
Suppose that there are only finitely many prime numbers, say $q_1,\ldots,q_n$,
and consider the number 
%% you can display equations like this ...
\[
N=1+q_1q_2\ldots q_n.
\]
Observe that $N>q_i$ for all $i$, so $N$ is not prime. It is therefore divisible by a prime,
say $q_j$. Then, however, $q_j$ divides $q_1\ldots q_n$ and $N=1+q_1\ldots q_n$,
so it therefore divides the difference $(1+q_1\ldots q_n)-(q_1\ldots q_n)=1$.
This contradicts the fact that $q_j$ is prime.
$\qed$
%% the user-defined \qed command produces a nice box to signal the end of a solution.

%% there are a few standard LaTeX commands in this document, for example:
%% \to
%% \tau, \rho
%% you will find many more standard commands on the links on Blackboard

\end{document}
